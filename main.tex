\documentclass[reqno]{amsart}
\usepackage{amsmath,amssymb,amsthm,amsfonts}
\usepackage{physics}
\usepackage{graphicx}
\usepackage[hidelinks]{hyperref}
\usepackage{babel}
\usepackage{datetime}
\usepackage{ragged2e}
\usepackage{siunitx}

% ----------------------------------------------------------------
\let\oldtocsection=\tocsection

\let\oldtocsubsection=\tocsubsection

\let\oldtocsubsubsection=\tocsubsubsection

\renewcommand{\tocsection}[2]{\hspace{0em}\oldtocsection{#1}{#2}}
\renewcommand{\tocsubsection}[2]{\hspace{1em}\oldtocsubsection{#1}{#2}}
\renewcommand{\tocsubsubsection}[2]{\hspace{2em}\oldtocsubsubsection{#1}{#2}}
\makeatletter
\renewcommand\subsection{\@startsection{subsection}{2}%
  \z@{.5\linespacing\@plus.7\linespacing}{-.5em}%
  {\normalfont\scshape\justify}}
\makeatother
% ----------------------------------------------------------------

\newdateformat{monthyeardate}{%
  \monthname[\THEMONTH] \THEYEAR}
\numberwithin{equation}{section}
\numberwithin{figure}{section}
\tolerance=1
\emergencystretch=\maxdimen
\hyphenpenalty=10000
\hbadness=10000
\setcounter{tocdepth}{3}

\title{Ground State Energy Simulations Using Monte Carlo Random Walk}
\author[Elliott Ashby]{Elliott Ashby \\ Physics and Astronomy \\ University of Southamton}
\date{\monthyeardate\today}
% ----------------------------------------------------------------

\begin{document}
\begin{abstract}
    Placeholder for abstract.
\end{abstract}
\maketitle
\tableofcontents
\newpage
\section{Introduction}

% ----------------------- v TODO v ----------------------- % 

Overview of the challenges in solving the Schrödinger equation numerically
The connection between the imaginary-time Schrödinger equation and diffusion processes
Historical development of Monte Carlo methods in quantum mechanics
Brief comparison with alternative numerical approaches (matrix diagonalization, variational methods)
Motivation: why random walks are particularly well-suited for ground state energy calculations

% ----------------------- ^ TODO ^ ----------------------- %
\section{Method}
\subsection{Theoretical Foundation}

While the usage of Monte Carlo methods to simulate quantum mechanics may seem strange at first, looking at the time-dependent Schrödinger equation as a diffusion equation in imaginary time reveals the connection that makes it possible. Here is the time-dependent Schrödinger equation for a particle of mass $m$ in a potential $V(x)$:

\begin{equation}
    i\hbar\frac{\partial}{\partial t}\psi(x,t) = -\frac{\hbar^2}{2m}\frac{\partial^2}{\partial x^2}\psi(x,t) + V(x)\psi(x,t) = \hat{H}\psi(x,t)
\end{equation}

By using $\tau = it$, it becomes:

\begin{equation}
    \frac{\partial}{\partial \tau}\psi(x,\tau) = \frac{\hbar}{2m}\frac{\partial^2}{\partial x^2}\psi(x,\tau) - \frac{V(x)}{\hbar}\psi(x,\tau) = -\frac{\hat{H}}{\hbar}\psi(x,\tau)
    \label{eq:diffusion}
\end{equation}

This is a diffusion equation with a growth/decay term with a diffusion constant $D = \hbar/2m$ where the growth/decay term determines whether the population density of walkers at a given position increases or decreases. \\

The general solution to equation \ref{eq:diffusion} can be written using the eigenstates and eigenvalues of the Hamiltonian:

\begin{equation}
\psi(x,\tau) = \sum_n c_n \phi_n(x)e^{-E_n\tau/\hbar}
\end{equation}

where $\phi_n(x)$ are the eigenfunctions of the Hamiltonian with corresponding energies $E_n$, and $c_n$ represents the projection of the initial wavefunction onto each eigenstate. If we consider the case where $\tau \to \infty$, all terms decay exponentially, but the ground state decays the slowest:

\begin{equation}
\lim_{\tau \to \infty}\psi(x,\tau) = c_0\phi_0(x)e^{-E_0\tau/\hbar}
\end{equation}

If we then were able to simulate the diffusion process for a sufficiently long time, then the distribution of our random walkers would approach the ground state wavefunction $\phi_0(x)$, so the decay rate of the walker population would be the ground state energy $E_0$.

\subsection{Implementation Details}

In order to not have infinite computation time, we must discretize space and time. To do this we can use a lattice with spatial sites $j$ and temporal sites $n$. The probability $p(j,n)$ for a walker to be at lattice site $j$ after $n$ time steps satisfies:

\begin{equation}
p(j,n+1) = \frac{1}{2}[1-a(j)][p(j-1,n) + p(j+1,n)]
\end{equation}

and converting to continuous space and time with intervals $h_x$ and $h_\tau$:

\begin{equation}
p(x, \tau + h_\tau) = \frac{1}{2} \left[1 - a(x)\right] \left[p(x - h_x, \tau) + p(x + h_x, \tau)\right]
\end{equation}

Here, $a(x)=h_\tau V(x)$ and represents the probability of a walker being absorbed at position $x$. \\

So for $\tau \to \infty$, swapping $\psi(x,\tau)$ for $p(x,\tau)$ and defining $h_\tau = h_x^2$ we get the final form of the diffusion equation:

\begin{equation}
    \frac{\partial p}{\partial \tau} = \frac{1}{2}\frac{\partial^2 p}{\partial x^2} - V(x)p
    \label{eq:diffusionFinal}
\end{equation}

\subsection{Algorithm Variants}

Using the diffusion equation \ref{eq:diffusionFinal} we can derive a couple different ways to simulate the ground state energy of a quantum system. \\

The simplest method is the simple absorption method, where walkers are absorbed at the boundaries of the lattice. This method is simple to implement but can be computationally expensive for large systems. \\

A more efficient method is the diffusion Monte Carlo method, which uses branching to kill and birth new walkers. This method is much more efficient but can be more difficult to implement due to the branching process.

\subsubsection{Simple Absorption Method}

In the case for the simple absorption method, our potential is implemented as a infinite square well potential. The probability of a walker being absorbed at a given position is given by:

\begin{equation}
    a(x) = \begin{cases}
        0 & \text{if } x \in [-L/2, L/2] \\
        1 & \text{otherwise}
    \end{cases}
\end{equation}

where $L$ is the width of the well. \\

Using this potential, we can implement the diffusion equation as a random walk on a lattice where:

\begin{equation}
    P_{survival}(\tau) = \int_{-L/2}^{L/2} p(x, \tau)^2 dx \approx e^{-\lambda \tau}
\end{equation}

The algorithm is then as follows:

\fbox{\parbox{\textwidth}{
        Simple Absorption Method
        \begin{enumerate}
            \item Initialize the walker population at the centre of the lattice.
            \item For each walker, move it to a neighbouring lattice site with equal probability to move to any of the neighbouring sites.
            \item If the walker is at the boundary of the lattice ($x \geq L/2$), kill it and record the number of steps it took to reach the boundary.
            \item Repeat steps 2-3 for a maximum number of steps. If any of the walkers survive, mark them as surviving.
            \item Find the number of walkers that survive at each step and using a linear regression calculate the decay rate of the population $\lambda$.
            \item The ground state energy can then be approximated as $E_0 = \lambda (L/2)^2$.
        \end{enumerate}
}}


\subsubsection{Diffusion Monte Carlo with Branching}

The diffusion Monte Carlo with branching method is more complex than the simple absorption method and requires some additional theory.



% ----------------------- v TODO v ----------------------- %
\subsubsection{Importance Sampling Methods}

\subsubsection{Computing Expectation Values}

\section{Results}
\subsection{One-Dimensional Test Cases}
\section{Discussion}
\subsection{Parameter Optimization}
  Effects of lattice spacing, time step, and walker population
  Balancing computational cost and accuracy
  Handling statistical fluctuations through averaging techniques

\subsection{Method Strengths and Limitations}
  Robust performance in higher dimensions
  Challenges with excited states and nodal surfaces
  Comparison with traditional numerical methods
  Applicability to different potential types

\subsection{Advanced Applications}
  Extensions to field theory with probability
    \begin{equation}
    a[\phi] = h_\tau\left(v \sum_{j=1}^{N} V(\phi)_j E_s\right)
    \end{equation}
  Many-body quantum systems
  Handling sign problems for fermions

\section{Conclusion}
Summary of key findings and method effectiveness
Practical recommendations for implementation
Future directions for method improvement
Potential applications to more complex quantum systems

\section*{Appendix}
Detailed derivations of key equations
Pseudocode for the main algorithms
Benchmark data and supplementary results


\appendix
\section{Additional Citations}
18 Examples

6 for each: random walk, Monte Carlo, random numbers

3 in academia, 3 outside academia

\bibliographystyle{unsrtdin}
\bibliography{citations}
\end{document}
